\documentclass[UTF8]{book}
\usepackage{CTEX}
\usepackage{anyfontsize}
\usepackage{amssymb}
% \usepackage{boondox-cal}
\usepackage{mathrsfs}

\title{测度论与概率论基础}
\author{lw2333}

\begin{document}

\maketitle

\chapter{可测空间和可测映射}

\newpage
\section{集合及其运算}
\textbf{Definitions}

称任意一个非空集合 $ X $ 为空间. $ X $ 的子集称为这个空间的\textbf{集合}.
空集记为{ $ \varnothing $ }.  $ X $  的成员称为元素.

定义 $ X $ 上的实函数\par
 $$ 
    I_A = \left\{
    \begin{array}{l}
        1,x\in A, \\
        0,x\notin A
    \end{array} \right. \
 $$ \par
称为 $ A $ 的指示函数.\par
集合 $ A^c := \{x:x\notin A\} $ 称为集合 $ A $ 的余.\par
集合 $ A\backslash B:=\{x\in A\wedge x\notin B\} $ 称为集合 $ A $ 和 $ B $ 的差或真差.\par
集合 $ A\Delta B:=(A\backslash B) \bigcup\, (B\backslash A) $ 称为 $ A $ 和 $ B $ 的对称差.\par
设 $ \{A_n,n=1,2,\ldots\} $ .如果对每个 $ n=1,2,\ldots $ 有 $ A_n \in A_{n+1} $ 则称 $ A_n $ 为非降的, 记为 $ A_n\uparrow $ .
并称 $ \lim_{n\to \infty}A_n:=\bigcup_{\infty}A_n  $ 为它的极限.\par
类似地, 定义非增的, 并称它们的交为它的极限. \par

统称二者为单调序列. 因此, 单调序列总有极限. (为什么?)\par
因为 $ \{\,\bigcap_{\,k=n}^{\infty}A_k\}\uparrow $ ,  $ \{\,\bigcup_{\,k=n}^{\infty}A_k\}\downarrow $ \par
因而分别有极限\par
 $ \liminf_{n\to \infty} A_n := \bigcup_{n=1}^\infty \bigcap_{k=n}^\infty A_k $ \par
 $ \limsup_{n\to \infty} A_n := \bigcap_{n=1}^\infty \bigcup_{k=n}^\infty A_k $  \par
分别称作 $ A_n $ 的下极限和上极限\par
 $ x\in \liminf_{n\to \infty} A_n $ 意味着除去 $ A_n $ 的有限个集合外,  $ x $ 属于该序列的其他集合\par
 $ x\in \limsup_{n\to \infty} A_n $ 意味着 $ x $ 属于 $ A_n $ 中无穷多个集合\par

于是我们有\par
 $ \liminf_{n\to \infty} A_n\subseteq \limsup_{n\to \infty} A_n $ \par
如果二者相等, 认为 $ A_n $ 的极限存在, 记作 $ \lim_{n\to \infty}A_n $ 


\newpage
\section{集合系}
\textbf{Definitions}\par
\vspace*{12pt}

空间 $ X $  上由集合为元素组成的集合称作集合  $ X $  上的集合系, 常用  $ \mathscr{A, B\ldots} $ 等表示\par

\vspace*{12pt}

\textbf{$ \pi $ 系:}\par
\quad\quad如果 $ X $ 上的非空集合系 $ \mathscr{P} $ 对于集合交的运算是封闭的, 即:\par
\quad\quad$ A, B\in \mathscr{P}\longrightarrow A\,\bigcap B \in \mathscr{P} $ \par

\vspace*{12pt}

\textbf{半环:}\par
\quad\quad满足下列条件的  $ \pi $  系  $ \mathscr{Q} $  被称为半环:\par
\quad\quad $ \forall A,\,B \in \mathscr{Q} $ 且 $ B\subseteq A$, $\exists $有限个两两不交的 $\{C_k \in \mathscr{Q}, k=1,2,3\ldots\}$\par
\quad\quad使得 $A\backslash B = \bigcup_{k=1}^nC_k$\par

\vspace*{12pt}

\textbf{环:}\par
\quad\quad如果非空集合系 $\mathscr{R}$ 对并和差的运算是封闭的,即\par
\quad\quad$A, B\in \mathscr{R}\Longrightarrow  A\bigcap B, A-B\in \mathscr{R}$\par
\quad\quad 则称 $\mathscr{R}$ 为环\par

\vspace*{12pt}

\textbf{域:}\par
\quad\quad满足下列条件的 $\pi$ 系称为域:\par
\quad\quad$X\in \mathscr{A}$\par
\quad\quad$\forall\,A\in \mathscr{A}, \Longrightarrow A^c\in \mathscr{A}$\par
有的文献中,也把域叫做代数\par

\vspace*{12pt}

\textbf{命题 1.2.1}\par
半环是 $\pi$ 系,环必是半环,域必是环\par

\vspace*{12pt}


\newpage
\textbf{Definitions}\par

\vspace*{12pt}

一些可列封闭的集合系:\par

\textbf{单调系:}\par
\quad\quad如果对集合系 $\mathscr{M}$ 中的任何单调序列 $\{A_n,n=1,2,\ldots\}$ \par
\quad\quad 均有 $\lim_{n\to \infty} A_n\in \mathscr{M}$ 
则把 $\mathscr{M}$ 叫做单调系\par

\vspace*{12pt}

\textbf{$\lambda$ 系:} \par
\quad\quad集合系 $\mathscr{L}$ 称为 $\lambda$ 系,如果它满足条件:\par
\quad\quad$X\in \mathscr{L}$\par
\quad\quad$\forall \,(A,B \in \mathscr{L} ,\, B\subseteq A)\Longrightarrow A\backslash B \in \mathscr{L}$\par
\quad\quad$\forall\,(A_n\in \mathscr{L},\,A_n \uparrow) \Longrightarrow\bigcap_{n=1}^\infty A_n\in \mathscr{L}$\par

\vspace*{12pt}

\textbf{$\sigma$ 系:} \par
\quad\quad集合系 $\mathscr{F}$ 称为 $\sigma$ 系,如果它满足条件:\par
\quad\quad$X\in \mathscr{F}$\par
\quad\quad$\forall \,A\in \mathscr{F}\Longrightarrow A^c \in \mathscr{F}$\par
\quad\quad$\forall\,(A_n\in \mathscr{L},\,n=1,2,\ldots) \Longrightarrow\bigcap_{n=1}^\infty A_n\in \mathscr{F}$\par
也称 $\sigma$ 系为 $\sigma$ 代数。有两个特殊的 $\sigma$ 域,\{$\varnothing,\,X$\} 和 \{$A: \,A\subseteq X$\}\par

设 $\mathscr{F}$ 是一个 $\sigma$ 域,那么\par
$A_n\in \mathscr{F}\,n=1,2,\ldots\Longrightarrow \bigcap_{n=1}^\infty A_n = \{\bigcup_{n=1}^\infty A_n^c\}^c\in \mathscr{F} $\par
故它对可列交也是封闭的。故得到:\par
$A,\,B \in\mathscr{F}\Longrightarrow A \bigcap B = A\bigcap B \bigcap X\ldots$\par
因此, $\sigma$ 域是域\par

\vspace*{12pt}

\textbf{命题1.2.2}\par
\quad\quad $\lambda$ 系是单调系; $\sigma$ 域是 $\lambda$ 系\par

\vspace*{12pt}

总结以上,得到由宽到紧的关系:\par
$\pi$ 系 $\rightarrow$ 半环 $\rightarrow$ 环 $\rightarrow$ 域 $\rightarrow$ $\sigma$ 域\par
单调系 $\rightarrow$ $\lambda$系 $\rightarrow$ $\sigma$域\par

\vspace*{12pt}

一个非空集合 $X$ 和它上的 $\sigma $ 域 $(X,\mathscr{F})$ 称为可测空间\par
\newpage
\textbf{命题1.2.3}\par
\quad\quad 一个既是单调系又是域的集合系必是 $\sigma$ 域\par

\textbf{命题1.2.4}\par
\quad\quad 一个既 $\lambda$ 系又是 $\pi$ 的集合必是 $\sigma$ 域\par
\quad\quad 一个对可列并封闭的环是 $\sigma$ 环;一个包含 $X$ 的 $\sigma$ 环是 $\sigma$ 域
\newpage

\section{$\sigma$ 域的生成}
\textbf{定义 1.3.1}\par
\quad\quad形式化语言略去. 由集合系 $\mathscr{E}$ 生成的环 $\mathscr{L}$ 就是包含 $\mathscr{E}$ 的最小环\par \quad\quad其他数据结构同理\par
\textbf{命题 1.3.1}\par
\quad\quad 由任意集合系生成的环、单调系、lambda系和sigma域均存在\par

\vspace*{12pt}

\textbf{THM 1.3.2}\par
\quad\quad 把由集合生成的环、单调系、$\lambda$ 系、$\sigma$ 域分别记作 $r, m, l, \sigma$\par
\quad\quad 半环 $\mathscr{R}$ 生成的环\par
\quad\quad $r(\mathscr{R}) = \bigcup_{n=1}^\infty\{\bigcup_{k=1}^nA_k:\{A_k\in \mathscr{R},k=1,2,\ldots \}{A_i\bigcap A_j = \varnothing}\}$\par
\textbf{THM 1.3.3}\par
\quad\quad 如果 $\mathscr{A}$ 是域,则 $\sigma(\mathscr{A}) = m(\mathscr{A})$\par
\textbf{推论 1.3.4}\par
\quad\quad 如果 $\mathscr{A}$ 是域,$\mathscr{M}$ 是单调系,则\par
\quad\quad $\mathscr{A}\subseteq \mathscr{M}\Longleftarrow\sigma(\mathscr{A})\subseteq\mathscr{M}$

\textbf{THM 1.3.5}\par
\quad\quad 如果 $\mathscr{P}$ 是 $\pi$ 系,则 $\sigma(\mathscr{P}) = l(\mathscr{P})$\par
\textbf{推论 1.3.6}\par
\quad\quad 如果 $\mathscr{P}$ 是 $\pi$ 系, $\lambda$


\vspace*{12pt}

\end{document}